\documentclass[11pt,letterpaper]{article}
\usepackage[latin1]{inputenc}
\usepackage[left=3.00cm, right=3.00cm, top=3.00cm, bottom=3.00cm]{geometry}
\usepackage{amsmath}
\usepackage{amsthm}
\usepackage{fancyhdr}
\pagestyle{fancy}

\usepackage{mathtools}
\DeclarePairedDelimiter\ceil{\lceil}{\rceil}
\DeclarePairedDelimiter\floor{\lfloor}{\rfloor}

\usepackage{color}
\usepackage{listings}
\usepackage{caption}

\usepackage{graphicx}

\author{Simon Zheng\\260744353}
\title{Homework 1}
\date{October 5$^{\textnormal{th}}$, 2017}
\lhead{COMP 273}
\rhead{Introduction to Computer Systems}

\begin{document}
	\maketitle
	\thispagestyle{fancy}
	
	\section{}
	Assume numbers are in decimal/base 10 unless indicated as otherwise, or written in their form (0000 0000 ... for binary and 0x... or with ABCDEF for hexadecimal).
	
		\subparagraph{.1}
		We keep dividing by 2 until we get 0 and all the remainders are the binary number digits (or rather, bits) in order.
		\begin{align*}
		741 / 2 &= 370&\textnormal{with remainder 1}\\
		370 / 2 &= 185&\textnormal{with remainder 0}\\
		185 / 2 &=  92&\textnormal{with remainder 1}\\
		 92 / 2 &=  46&\textnormal{with remainder 0}\\
		 46 / 2 &=  23&\textnormal{with remainder 0}\\
		 23 / 2 &=  11&\textnormal{with remainder 1}\\
		 11 / 2 &=   5&\textnormal{with remainder 1}\\
		  5 / 2 &=   2&\textnormal{with remainder 1}\\
		  2 / 2 &=   1&\textnormal{with remainder 0}\\
		  1 / 2 &=   0&\textnormal{with remainder 1}\\
		\\
		&&(10\ 1110\ 0101)_2
		\end{align*}
		
		\subparagraph{.2}
		As 16 is a multiple of 2, converting from binary to hexadecimal can be very convenient (1 digit base 16 = 4 bits), but here using the same technique to convert decimal to hexadecimal than we did with binary is sufficient. Instead of dividing by 2 we divide by 16. Thus the remainder can be any digit in the range 1, 2, ..., E, F which we use to get each digit of the converted number in base 16 in order, just like for binary.
		\begin{align*}
		741 / 16 &= 46&\textnormal{with remainder 5}\\
		 46 / 16 &=  2&\textnormal{with remainder E}\\
		  2 / 16 &=  0&\textnormal{with remainder 2}\\
		\\
		&&(2E5)_{16}\textnormal{ or 0x2e5}
		\end{align*}
		
		\subparagraph{.3}
		For fractional numbers, we calculate the integer part as previously done, and the fractional part separately. For the latter, instead of dividing we multiply by 2 and we always take the integer part (1 or 0) as a bit in order. Though when multiplying each time, we only keep the fractional part.
		
		We then just put the converted integer part to the left of the point and the fractional part to the right, exactly like the original number.
		\begin{align*}
		0.3515625 * 2 &= 0.703125&\textnormal{with integer part 0}\\
		0.703125  * 2 &= 1.40625 &\textnormal{with integer part 1}\\
		0.40625   * 2 &= 0.8125  &\textnormal{with integer part 0}\\
		0.8125    * 2 &= 1.625   &\textnormal{with integer part 1}\\
		0.625     * 2 &= 1.25    &\textnormal{with integer part 1}\\
		0.25      * 2 &= 0.5     &\textnormal{with integer part 0}\\
		0.5       * 2 &= 1       &\textnormal{with integer part 1}\\
		\end{align*}
		The integer part is obviously just 1. Thus, we get $(1.0101101)_2$.
		
		\subparagraph{.4}
		Same thing but with 16. Integer part 1 is still 1, fractional is:
		\begin{align*}
		0.3515625 * 16 &= 5.625&\textnormal{with integer part }&5\\
		5.625     * 16 &= 10   &\textnormal{with integer part }&10\\
		\\
		(1.5A)_{16}&&(\ (10)_{10} = (A)_{16}\ )&\\
		\end{align*}
		
		\subparagraph{.5}
		In base 2, going through the bits from right to left, each bit represents a power of 2 in increasing order. Whether that position has a 1 or a 0 represent whether you "count" that multiple or not (or rather, multiply each bit with their positional value in base 10) and then add all of them together to get the number in base ten. This is just like how in base 10, rightmost digit position is 1, then 10, then 100, each being to next biggest power of the base. So the digit occupying the position is the number of times that values is counted (adding all the digits multiplied by their "positional value" or power of 10). This comes from positional notation.
		Thus,
		\begin{align*}
		(100\ 1101)_2 &= (1 \times 64) + (0 \times 32) + (0 \times 16) + (1 \times 8) + (1 \times 4) + (0 \times 2) + (1 \times 1)\\
		&= 77
		\end{align*}
		
		\subparagraph{.6}
		As said in 1.1.2, converting binary to hexadecimal can be very convenient, even though here going from decimal is also short. We apply the same principle as the previous conversion to decimal but here, instead of using all of the bits and adding them together, we can instead simply map each digit of the hexadecimal number to a nibble of the binary representation. Thus we can do the same operations as in the previous conversion, but instead we do it in base 16 in each nibble separately where they become a digit of the hexadecimal representation.
		\begin{align*}
		(0100\ 1101)_2 &= (\{1 \times 4\}\{(1 \times 1) + (1 \times 4) + (1 \times 8)\})_{16}\textnormal{ (using positional notation)}\\
		&= 4D
		\end{align*}
		Comparing with the first way of repeated division and taking the remainder for converting using the result from 1.1.5:
		\begin{align*}
		77 / 16 &= 4&\textnormal{with remainder\;\;\:}&(13)_{10} = (D)_{16}\\
		 4 / 16 &= 0&\textnormal{with remainder 4}&\\
		\\
		&&&(4D)_{16}
		\end{align*}
		The results match.
		
		\subparagraph{.7}
		Once again, just like we transformed dividing by two into multiplying by two for the fractional part to convert from decimal to binary, here we can also use positional notation to convert binary to decimal. Since we're working in fractional values, each position from left to right is a power of $2^{-i}$ where $i$ is the position from the point.
		\begin{align*}
		(0.101011)_2 &= 0 + (2^{-1}) + (2^{-3}) + (2^{-5}) + (2^{-6})\\
		&= (0.671875)_{10}
		\end{align*}
		
		\subparagraph{.8}
		For hexadecimal we just do the same thing as usual when converting from binary.
		\begin{align*}
		(0.101011)_2 &= 0.\{(1010)_2\}\{(1100)_2\}\\
		&= 0.\{(1 \times 8) + (0 \times 4) + (1 \times 2) + (0 \times 1)\}\{(1 \times 8) + (1 \times 4)\}\\
		&= 0.AC
		\end{align*}
		As 0 is still 0 in hexadecimal.
		
		\subparagraph{.9}
		We can just do the opposite of 1.1.6, and it is very easy using positional notation as each digit only goes from 1 to F so 16 possibilities, 4 bits.
		\begin{align*}
		(F)_{16} = (1111)_2\\
		(0)_{16} = (0000)_2\\
		(D)_{16} = (1101)_2\\
		\end{align*}
		Thus $(F00D)_{16} = (1111\ 0000\ 0000\ 1101)_2$.
		
		\subparagraph{.10}
		Here again we simply make use of positional notation except in base 16 (powers of 16) to get the delicious conversion:
		\begin{align*}
		(F00D)_16 &= ((F)_{16} \times 4096) + ((0)_{16} \times 256) + ((0)_{16} \times 16) + ((D)_{16} \times 1)\\
		&= (15 \times 4096) + (0 \times 256) + (0 \times 16) + (13 \times 1)\\
		&= 61 453
		\end{align*}
		
		\subparagraph{.11}
		
		\subparagraph{.12}
		
	\section{}
	
		\subparagraph{.1}
		Number is positive so sign bit is 0.
		Integer part is just 1 so $1 \times 2^0$, and adding the bias (+127) to the exponent 0, the exponent bits are $127 = (0111\ 1111)_{2}$.
		For the fractional part,
		\begin{align*}
		1.00001\\
		0.00001 * 2 &= 0.00002&\textnormal{with integer part 0}\\
		&&...\textnormal{ (14 more zeroes)}\\
		0.32768 * 2 &= 0.65536&\textnormal{with integer part 0}\\
		0.65536 * 2 &= 1.31072&\textnormal{with integer part 1}\\
		0.31072 * 2 &= 0.62144&\textnormal{with integer part 0}\\
		0.62144 * 2 &= 1.24288&\textnormal{with integer part 1}\\
		0.24288 * 2 &= 0.48576&\textnormal{with integer part 0}\\
		0.48576 * 2 &= 0.97152&\textnormal{with integer part 0}\\
		0.97152 * 2 &= 1.94304&\textnormal{with integer part 1}\\
		0.94304 * 2 &= 1.88608&\textnormal{with integer part 1}\\
		0.88608 * 2 &= 1.77216&\textnormal{with integer part 1}\\
		&&...\textnormal{ (we stop as we've already passed 23 mantissa bits)}\\
		0. * 2 &= .&\textnormal{with integer part 0}\\
		0. * 2 &= .&\textnormal{with integer part 1}\\
		\\
		&&(0000\ 0000\ 0000\ 0000\ 1010\ 0111)_2
		\end{align*}
		Thus the IEEE floating point number representation with 1 sign bit, 8 exponent bits and 23 mantissa bits is:
		\begin{center}
		0 $\|$ 0111 1111 $\|$ 0000 0000 0000 0000 1010 0111
		\end{center}
		
\end{document}
